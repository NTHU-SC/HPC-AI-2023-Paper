\section{Introduction}
\label{sec:intro}

\IEEEPARstart{T}he 2023 High-Performance Computing and Artificial Intelligence (HPC-AI) Asia-Pacific Competition was organized by the HPC-AI Advisory Council and the National Supercomputing Centre (NSCC) of Singapore. The competition attracted 22 outstanding teams from 11 countries and regions. The competition spanned from May 19 to November 16, 2023. The HPCAI competition for this year announced two distinct challenges: the Climate Prediction Model MPAS and the Large Language Model Bloom. The first challenge, MPAS, is restricted to be completed using only CPUs, while the second challenge, Bloom, is constrained to utilize GPUs. The combination of scientific computing with AI challenges aligns well with the spirit of HPCAI.

Our goal in this competition was straightforward - to accelerate as much as possible. While, since code modification was not allowed in this competition, initially, we faced challenges in achieving performance improvements. Consequently, we directed our efforts towards developing an optimized application and creating an environment with optimized libraries.

In terms of the final results, although we didn't see significant improvements in performance, we tried various optimization methods. This paper will outline all the approaches we attempted, the challenges we faced, and our observations and solutions for overcoming difficulties.